\Chapter{Bevezetés}

A természetes nyelvfeldolgozás az informatika egyik talán legkomplexebb feladatköre. Ennek egyik oka, hogy az emberi nyelv és annak kialakulása szorosan összefügg az emberi aggyal és annak evolúciójával, melyet még a mai napig se sikerült teljesen feltérképeznünk és megértenünk. Nyelvünk értő használata egyike azon utolsó problémaköröknek, amiket a számítógépek eddig nem voltak képesek még megközelítőleg se megfelelően teljesíteni, hiszen akár már egy egyszerű mondat feldolgozásához, kontextusban való elhelyezéséhez vagy akár kibővítéséhez is óriási méretű szabályhalmazokra és számítási teljesítményre van szükség. 

Az utóbbi időkben azonban jelentős sikereket tudtak elérni a gépek más területeken. Egyre elterjedtebbé váltak a különböző objektumfelismerő algoritmusok, melyek akár alacsony minőségű képekből is képesek felismerni alakzatokat, formákat és mindezt közel valós időben mozgás közben is. Ezen algoritmusoknak már számos vállalati és állami felhasználása is van. 

Továbbá megjelentek különféle képgeneráló algoritmusok, melyek generatív versengő hálózatok(GAN hálózat) segítségével művészi minőségű képeket tudnak generálni. Számos weboldal készült már, ahol pár sor szöveg megadása után a szerver generál egy képet a megadott szöveg alapján és megjeleníti azt tetszőleges minőségben. Ez a megoldás már szövegfelismerést is tartalmaz, valamint a szöveg egyes részeihez csatolt képi alakzatokat, melyek segítségével mondjuk egy GAN hálózat megkonstruálhatja a kívánt képet, akár csak egy igazi művész.

De vajon képesek-e a gépek magát a szöveget elkészíteni egy ilyen képgenerálóhoz? Képesek-e írói minőségű vagy kultúrális igényű szövegeket készíteni? Tudnak-e kérdéseket megfogalmazni vagy éppen válaszokat? Ezekre a kérdésekre fogom keresni a választ diplomamunkámban. Bemutatom továbbá egy példaprogramon keresztül a jelenleg rendelkezésünkre álló fejlesztői környezetet és könyvtárakat, valamint a szöveggenerálás során jelenleg használt legfejlettebb algoritmusokat is.


