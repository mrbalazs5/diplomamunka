\documentclass[12pt,a4paper]{report}

\usepackage{styles/dolgozat}

\usepackage{listings}
\usepackage{styles/cpp}
\usepackage{styles/python}

\usepackage{hyperref}

\begin{document}

\Chapter{Összegzés}

Mint láthattuk dolgozatomban is, a természetes nyelvfeldolgozás virágkorát éli napjainkban. Egymás után jelennek meg olyan alkalmazások, chatbotok melyek szinte tökéletesen képesek megérteni az emberi nyelveket, képesek szöveges feladatokat megoldani, kódokat generálni, cikkeket értelmezni és írni, szöveges hibákat keresni vagy ahogy láthattuk is immáron képesek értelmes kérdéseket is generálni nagyon jó eredményekkel. 

Úgy gondolom, hogy alkalmazásunk remekül beilleszthető ezen megoldások mellé, de nyilván van rajta még javítani való. Láthattuk, hogy a ChatGPT-vel összehasonlítva azért teljesítményben van még hova fejlődnie, de kisebb mérete és jobb hardvereken való gyorsabb teljesítménye, így is számos plusz pontot ad programunknak.

Ahol lehetne még fejleszteni az alkalmazást az mindenképp a tudásbázisa. Amellett, hogy többnyelvűvé is lehetne tenni, növelni lehetne a tudását egy jobban szűrt, többféle kérdés és kontextustípust is tartalmazó tanítóhalmazzal. Továbbá nem ártana a hardveres erőforrásokon is javítani és egy még nagyobb, GPU-t vagy TPU-t kihasználó felhőszolgáltatáson taníttatni és működtetni, ahol kihozhatnánk belőle a maximumot. Ez természetesen a többi neurális hálózatokat használó alkalmazás problémája is, hiszen a hálózat méretét növelve egyre jelentősebb lesz a teljesítményigény is, így többnyire nagyobb cégek tudnak ezekhez megfelelő erőforrásokat biztosítani.

Ehhez kapcsolódóan fontos kérdés lehet még ezen mély tanulás alapú megoldások, illetve magának az informatikai szektornak a teljesítményigénye, hiszen egy-egy komolyabb modell tanítása óriási költségekkel járhat, ami a hibajavítást is megnehezítheti, hiszen nagyobb volumenű hibáknál sokszor nincs lehetőség az egész hálózatot újratanítani, mivel az akár hónapokig is eltarthat és rengeteg erőforrást felemészthet. Természetesen a technológia fejlődése itt is megoldhatja a problémákat, de a jövőben erre mindenképp oda kell majd figyelni.

Végül fontosnak tartom még megemlíteni az ilyen neurális hálózat alapú, NLP megoldások társadalmunkat is érintő következményeit. Világosan látszik, hogy közelítünk egy olyan ponthoz, ahol a 20. században felvázolt gépi intelligenciát behatároló korlátokat kezdjük átlépni. Sorra jelennek meg olyan alkalmazások a terület kutatásai alapján, melyek képesek megoldani olyan feladatokat, amiket eddig csak emberek, külön képesítéssel tudtak elvégezni, azonban egyre többször találkozhatunk azzal is, hogy a gép szintaktikailag jól oldotta meg a feladatot, de valójában helytelen eredményeket generált. Úgy gondolom ezen a területen is rengeteget kell még fejlődnie a tudománynak és a neurális hálózatokat önmagyarázóvá és könnyebben karbantarthatóvá kell tenni, más különben eláraszthatják az internetet hamis, pontatlan információkkal, amik egy láncreakció folytán a később modellek eredményeit is ronthatják, illetve társadalmunkra is negatív hatással lehetnek.

Úgy vélem dolgozatom jó táptalaj lehet ezen megoldások elkészítéséhez a jövőben és én is törekedni fogok a terület népszerűsítésére és kutatására az elkövetkezendőkben.

\end{document}