\documentclass[12pt,a4paper]{report}

\usepackage{styles/dolgozat}

\usepackage{listings}
\usepackage{styles/cpp}
\usepackage{styles/python}

\usepackage{hyperref}

\begin{document}

\Chapter{Summary}

As we saw in my thesis, natural language processing is currently in its heyday. Applications and chatbots appear one after another that are able to understand human languages almost perfectly, are able to solve text tasks, generate codes, interpret and write articles, search for text errors or, as we have seen, are now able to generate meaningful questions with very good results.

I think that our application can be perfectly integrated with these solutions, but there is obviously room for improvement. We could see that, compared to ChatGPT, it still has room for improvement in terms of performance, but its smaller size and faster performance on better hardware still give our program many plus points.

Where the application could still be improved is definitely its knowledge base. In addition to making it multilingual, it's knowledge could be increased with a more filtered training set containing several types of questions and contexts. Furthermore, it would not hurt to improve the hardware resources and have it taught and operated on an even larger cloud service that uses GPU or TPU, where we could get the most out of it. This is, of course, also the problem of other applications using neural networks, since increasing the size of the network will also increase the performance requirements, so mostly larger companies can provide adequate resources for these.

Relatedly, these deep learning-based solutions and the performance requirements of the IT sector itself can also be an important issue, since training a serious model can entail enormous costs, which can also make error correction difficult, since it is often not possible to retrain the entire network in the case of large-volume errors, since it can take months and consume a lot of resources. Of course, the development of technology can solve the problems here as well, but in the future we will definitely have to pay attention to this.

Finally, I consider it important to mention the consequences of such neural network-based NLP solutions that also affect our society. It is clear that we are approaching a point where we are starting to cross the boundaries of machine intelligence outlined in the 20th century. Applications based on research in the field are appearing one after another, which are able to solve tasks that until now only humans, with special qualifications, could do, but we can also come across more and more that the machine solved the task syntactically well, but actually generated incorrect results. I think that science still needs to improve a lot in this area and neural networks need to be made self-explanatory and easier to maintain, otherwise they can flood the internet with false and inaccurate information, which, due to a chain reaction, can worsen the results of later models and have a negative impact on our society. .

I believe that my thesis can be a good breeding ground for preparing these solutions in the future, and I will also strive to popularize and research the field in the future.

\end{document}